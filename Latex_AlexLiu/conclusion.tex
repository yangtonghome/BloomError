\section{Conclusion}
\label{sec:conclusion}
%\vspace{-0.2in}
In this paper, we discuss the controversy of the formula of Bloom Filter. Three formulas of false positive probability are presented: Bloom's formula, Bose's formula, and Christensen's formula. There is an error in the deduction process of Bloom's formula, and a minor error in Bose's formula. Christensen's formula is correct, but the false positive must be caculated using an iterative table-based algorithm with a time complexity of $O(knm)$. What is worse, it cannot deduce the optimal value of $k$.
%Christensen's formula is correct, but can only compute the false positive when $m$ is small, it cannot deduce the optimal value of $k$.


To compute the optimal value of $k$, we use information and entropy theory to deduce the exact formula of $k$.
To compute the false positive probability, 1) For small $m$, Christensen's formula can be used; 2) For large $m$, we propose a new upper bound which is much more accurate than state-of-the-art. Fortunately, when $m$ is infinitely large, our upper bound becomes the same as the lower bound, which is Bloom's formula.
Besides, we derive the bounds of correct rate of counting Bloom filters through our proposed formulas about Bloom filters. 
All the implementation source code is made publicly available anonymously at GitHub \cite{opensource}.
%we prove a limit form of false positive probability using left and right limit, and find out that it is same as Bloom's. 
%3) We also show that the error is negligible when $m$ is not large using two simple examples.
\newpage