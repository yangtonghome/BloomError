\presec \section{Prior Art on BF False Positives} \postsec \label{sec:priorarts}
%
In this section, we review the prior art on calculating the false positive probability of Bloom filters.
%
Table \ref{table:symbols} summarizes the notations used in this paper.

\begin{table} [htbp]
\vspace{0in}
\caption{Notations and abbreviation used in this paper}
\centering
\label{table:symbols}
\begin{tabular}{| c | l |}
\hline Symbol & Description \\
\hline $\mathcal{S}$ & Set of elements\\
\hline $m$  & BF size\\
\hline $n$  & Number of elements in $\mathcal{S}$\\
\hline $k$  & Number of hash functions \\
\hline $k^*$  & Optimal number of hash functions \\
\hline $f$  & False positive probability \\
\hline $f_{bloom}$ & False positive probability calculated by Bloom \\
\hline $f_{bose}$ & False positive probability calculated by Bose \\
\hline $f_{christ}$ & False positive probability calculated by Christensen \\
\hline $f_{true}$        & True false positive probability of BF\\
\hline FP        & false positive\\
\hline BF        & Bloom filter\\
\hline
\end{tabular}
\end{table}

\presub \subsection{Bloom's False Positive Formula} \postsub
%
In 1970, Bloom calculated the false positive probability of a Bloom filter as follows \cite{BF1970}.
%
Given a set $\mathcal{S}$ of elements, let $n$ be the number of elements in $\mathcal{S}$, $k$ be the number of hash functions, and $m$ be the number of bits in the Bloom filter $A$ constructed from set $\mathcal{S}$.
%
In querying an element $x$, the false positive happens when the Bloom filter reports that $x \in \mathcal{S}$ (\ie, $A[h_i(x)]=1$ holds for each $1 \leqslant i \leqslant k$), but actually $x \notin \mathcal{S}$.
%
Consider an arbitrary bit $A[b]$ in $A$.
%
For any element in $\mathcal{S}$ and any hash function $h_i$ ($1 \leqslant i \leqslant k$), the probability that this element is not hashed to bit $A[b]$ by $h_i$ is $1-1/m$.
%
As $\mathcal{S}$ has $n$ elements and each element is hashed $k$ times, the probability of $A[b]=0$ is $p'$:
%
\begin{equation}
p'=\left(1-\dfrac{1}{m}\right)^{kn}
\label{p'form}
\end{equation}
%
Thus, the probability of $A[b]=1$ is $1-(1-1/m)^{kn}$.
%
For any element $x \notin \mathcal{S}$, the probability that the false positive happens for $x$, \ie, the probability of $A[h_1(y)] \wedge A[h_2(y)] \wedge \cdots, A[h_{k}(y)]=1$, is calculated as follows:
%
\begin{equation}
\label{fBloom}
f_{bloom} = \left(1 - \left(1 - \frac{1}{m}\right)^{kn}\right)^k
\end{equation}
%
This formula can be approximated by $(1-e^{\frac{-nk}{m}})^k$, based on which we can trade off between space indicated by $m$ and time indicated by $k$.

\presub
\subsection{Bose's Derivation} \postsub
%
In 2008, Bose \etal pointed out that the last step in Bloom's derivation is flawed because for any element $x \notin \mathcal{S}$, which is hashed into $k$ bits $A[h_1(x)], A[h_2(x)], \cdots, A[h_k(x)]$, the $k$ events $A[h_1(x)]=1, A[h_2(x)]=1, \cdots, A[h_k(x)]=1$ are not actually independent \cite{bose2008false}.
%
Although for each bit $A[h_i(x)]$ ($1 \leqslant i \leqslant k$), after inserting $n$ elements into array $A$, the probability of $A[h_i(x)]=1$ is $1-(1-1/m)^{kn}$, for the probability of $A[h_1(y)] \wedge A[h_2(y)] \wedge \cdots, A[h_{k}(y)]=1$ to be $(1-(1-1/m)^{kn})^k$, the $k$ events $A[h_1(x)]=1, A[h_2(x)]=1, \cdots, A[h_k(x)]=1$ need to be independent.
%
We now analyze the reason that the $k$ events $A[h_1(x)]=1, A[h_2(x)]=1, \cdots, A[h_k(x)]=1$ are not independent.
%
Let us first consider the probability of $A[h_1(x)]=1$, which is $1-(1-1/m)^{kn}$.
%
Second, we consider the probability of $A[h_2(x)]=1$ based on the condition that $A[h_1(x)]=1$.
%
There are two possibilities for $h_2(x)$: (1) $h_2(x) \neq h_1(x)$, and (2) $h_2(x) = h_1(x)$.
%
For the first case, the probability of $A[h_2(x)]=1$ is $1-(1-1/(m-1))^{kn}$.
%
For the second case, the probability of $A[h_2(x)]=1$ is 1.
%
Similarly, we can analyze the $A[h_3(x)]=1$ based on the condition that $A[h_2(x)]=1$ and $A[h_1(x)]=1$, etc.
%
Observing the dependency of the $k$ events $A[h_1(x)]=1, A[h_2(x)]=1, \cdots, A[h_k(x)]=1$, Bose \etal derive the following false positive formula:

\begin{equation}
\label{fBose}
f_{bose}=\dfrac{1}{m^{k(n+1)}}  \sum\limits_{i=1}^{m}i^k i! \binom{i}{m}  \left(\dfrac{1}{i!} \sum\limits_{j=0}^{i} (-1)^j \binom{i}{j}j^{kn}\right)
\end{equation}

%%
Bose \etal derived asymptotically closed forms for the upper and lower bounds of the above formula:
%
\begin{equation}
\label{fBound}
f_{bloom} < f_{bose} \leqslant f_{bloom} \times \left(1 + \mathcal{O}\left(\frac{k}{p} \sqrt{\frac{\ln m - k\ln p}{m}}\right)\right)
\end{equation}
where $p = 1 - p'=1-\left(1-\dfrac{1}{m}\right)^{kn}$.
%
These bounds hold under the condition that
\begin{equation}
\label{boundCon}
\frac{k}{p} \sqrt{\frac{\ln m - k\ln p}{m}} \leqslant c
\end{equation}
for some constant $c < 1$.
%
Bose \etal further showed that for $k \geqslant 2$, $f_{bose}$ is strictly larger than $f_{bloom}$, and the lower bound converges to $f_{bloom}$ when $m$ becomes infinitely large.

%p = 1 - p'

\presub
\subsection{Christensen's Derivation} \postsub
%
In 2010, Christensen \etal pointed out that Bose's formula has a mistake that the term $(-1)^j$ should be $(-1)^{i-j}$, but the lower and upper bounds in Eq. \ref{fBound} are correct. 
%
Christensen \etal derived the finally correct false positive formula for Bloom filters as follows:
%
\begin{equation}
\label{fChristen}
f_{christ}=\dfrac{m!}{m^{k(n+1)}}\sum\limits_{i=1}^{m} \sum\limits_{j=0}^{i} (-1)^{i-j}  \dfrac{j^{kn}i^k}{(m-i)!j!(i-j)!}
\end{equation}

%%
Although Christensen's formula is perfectly accurate, it is not much useful in practice.
%
First, given the Bloom filter parameters $n$, $m$, and $f$, it is difficult to calculate the optimal $k$ value as Christensen's formula does not give a closed form expression for calculating the optimal $k$.
%
Second, given the Bloom filter parameters $n$, $m$, and $k$, the algorithm by Christensen \etal takes $\mathcal{O}(knm)$ time to calculate the false positive probability $f$, which is time-consuming.