\presec
\newpage
%\vspace{-0.4in}
%\vspace{-0.07in}
\section{Correct Rate of Counting Bloom Filters}\postsec
\label{sec:cbfcr}
%\vspace{-0.03in}
The counting Bloom filter (CBF) \cite{cbf}, one widely used variant of standard Bloom filter, replaces each bit with one counter, supporting estimating the frequency of each element in a multiset. 
Specifically, given a multiset $\mathcal{S}$ of $n$ distinct elements with their corresponding frequencies, we create a counter array $A$ of length $m$ as follows. 
First, we initialize each counter of $A$ to 0, then for each element $x \in \mathcal{S}$, we use $k$ hash functions to compute $k$ hash values: $h_1(x), h_2(x),..., h_{k}(x)$ where each hash value is in the range $[1, m]$.
%
Second, for each $1 \leqslant i \leqslant k$, we let $A[h_i(x)]=A[h_i(x)] +1$.
%
Let $f_x$ be the frequency of element $x$ in multiset $\mathcal{S}$. 
% and its frequency $f_x$
Therefore, the step of $A[h_i(x)]=A[h_i(x)] +1~\text{for each}~1 \leqslant i \leqslant k$ will occur $f_x$ times. 
%
The resulting counter array $A$ is called the CBF for multiset $\mathcal{S}$.
%
To query the frequency of an element $y$ in multiset $\mathcal{S}$, we first use the same $k$ hash functions to compute $k$ hash values: $h_1(y), h_2(y),..., h_{k}(y)$.
%
Second, we report the minimum value of the $k$ counters: $A[h_1(x)], A[h_2(x)], ..., A[h_k(x)]$ as the estimated frequency of this element. 
%Second, we check whether the corresponding $k$ bits in $A$ are all $1s$ (\ie, whether $A[h_1(y)] \wedge A[h_2(y)] \wedge \cdots, A[h_{k}(y)]=1$ holds); if yes, then $y \in S$ may probably hold and we can further check whether $y \in S$; if no, then $y \in S$ definitely does not hold.
%
Obviously, the estimated frequency reported by the CBF is always  larger than or equal to the real frequency for any element in multiset $\mathcal{S}$. 
The case that the estimated frequency from the CBF is equal to the real frequency for one element is called the correct case. 
The probability of such case happening is called the correct rate of the CBF ($\mathcal{C}_r$). 
%%
%The FP probability $f$ can be calculated from $n$, $k$, and $m$.


The calculation of the correct rate of CBFs can benefit from our derivation of the false positive probability of standard Bloom filter.
In querying an element $x$, the correct case happens when there exists at least one hashed counter (among $A[h_1(x)], A[h_2(x)], ..., A[h_k(x)]$) that is not hashed by any elements in multiset $\mathcal{S} \setminus \{x \cdot f_x\}$. 
The contrapositive is that the correct case does not happen when all the $k$ hashed counters are also hashed by some elements in multiset $\mathcal{S} \setminus \{x \cdot f_x\}$. 
%the false positive happens when the Bloom filter reports that $y \in \mathcal{S}$ (\ie, $A[h_i(x)]=1$ holds for each $1 \leqslant i \leqslant k$), but actually $x \notin \mathcal{S}$.
Consider an arbitrary counter $A[b]$ in $A$.
%
For any distinct element in $\mathcal{S} \setminus \{x \cdot f_x\}$ and any hash function $h_i$ ($1 \leqslant i \leqslant k$), the probability that this element is not hashed to counter $A[b]$ by $h_i$ is $1-1/m$.
%
As $\mathcal{S} \setminus \{x \cdot f_x\}$ has $n-1$ distinct elements and each distinct element is hashed $k$ times, the probability that $A[b]$ is not hashed by any element in multiset $\mathcal{S} \setminus \{x \cdot f_x\}$ is $p_c$:
%
\begin{equation}
p_c=\left(1-\dfrac{1}{m}\right)^{k(n-1)}
\label{pform_c}
\end{equation}
%
Thus, the probability that $A[b]$ is hashed by some elements in multiset $\mathcal{S} \setminus \{x \cdot f_x\}$ is $1-(1-1/m)^{k(n-1)}$. 
The probability that all the $k$ hashed counters are also hashed by some elements in multiset $\mathcal{S} \setminus \{x \cdot f_x\}$ is answered by $f_{true}$ with element number of $n-1$. 
We denote $f_{true}|_{n-1}$ as $f_{true}$ with element number of $n-1$, and get: 
\begin{equation}
1 - \mathcal{C}_r = f_{true}|_{n-1} \Rightarrow \mathcal{C}_r = 1 - f_{true}|_{n-1}
\end{equation}
Applying Eq. \ref{bounds}, we can get the upper and lower bounds of the $\mathcal{C}_r$ of CBFs: 
\begin{equation}
{\small
\begin{aligned}
\label{cbfbounds}
1 - \left( 1- \left( 1-\dfrac{k}{m} \right)^{n-1} \right) ^k < \mathcal{C}_r < 1 - \left( 1- \left( 1-\dfrac{1}{m} \right)^{(n-1)k} \right) ^k
\end{aligned}
}
\end{equation}

