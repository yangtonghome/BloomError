\begin{abstract}
Bloom Filter is a memory-efficient data structure that has already been applied to many fields. A Bloom Filter tells whether an element belongs to a set or not. An element not belonging to the set might be judged to be is called a false positive. The formula of the false positive probability is deduced in 1970~\cite{BF1970}, and appears in many papers. Two papers in 2008 \cite{bose2008false}
and 2010 \cite{ken2010false} argue that the formula is wrong and give new formulas. 
%A correct formula is indispensable for the various applications of Bloom Filters.
To settle this controversy, we deduce the false positive probability using \textit{entropy theory}, and make two conclusions: 1) we deduce the limit form of the false positive probability, and show that the error of the Bloom's formula is so small that it can be ignored when the size of Bloom Filters is large; 2) the exact formula of optimal number of hash functions is deduced, and is close to the formula in 1970.
%本文讨论关于BF的假阳性计算的争议,并且针对BF本身存在的K值较大和无法动态增长的问题,提出了两种解决方案
\end{abstract}