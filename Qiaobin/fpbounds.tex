\section{Asymptotic Form of the FP Probability}
\label{sec:limitf}
We will derive a new and very simple approach to computing asymptotic form for the FP probability of BFs. The new derivation is based on \textit{partitioned Bloom Filters} (pBF) that are used frequently to carry out parallel query. Its principle is simple: the BF is divided into $k$ even partitions, and each hash function only acts on one of the partitions, respectively. %With this structure Using this method, there is a characteristic/regulation in the partitioned Bloom Filter: each partition of the Bloom Filter has at least one 1, thus its entropy is not the maximum.
The probability that one bit of the BF array remains 0 after inserting $n$ elements in the BF becomes
\begin{equation}
p'_{partition} = \left( 1-\dfrac{k}{m} \right)^n
\label{equ:p'pBF}
\end{equation}
as now each hash maps into $\frac{m}{k}$ separate bits.

It is intuitive that the FP probability of partitioned BF is a littler bigger than that of BF. Unfortunately, there is no strict proof. Here we show one proof method, which is based on the following Lemma. 

\textbf{Lemma II:} For $m > 1 \;\;,   k >1, \;\; n > 1, \: m>k$,
\begin{equation}
\label{theorem1}
\left( 1-\dfrac{k}{m} \right)  ^n <
\left( 1-\dfrac{1}{m} \right)  ^{kn}
\end{equation}

A similar but different inequality is shown in page 3 of the well known BF survey \cite{BFSurvey}. The survey did not give the proof, thus we give one here.

\begin{proof}
Let $g(k)=( 1-\frac{1}{m} )  ^k- ( 1-\frac{k}{m} )$. For $m>1$ and $k>1$, $g(k)$ is a continuous and derivable function, and we can obtain the following inequality in terms of its derivative:
%\begin{equation}
%g'(k)=\left( 1- \dfrac{1}{m}\right)^k \ln\left(1-\dfrac{1}{m}\right)+\dfrac{1}{m}
%\end{equation}
%Since $m > 1$, $k > 1$, based on the derivative, we have:
\begin{equation}
\begin{aligned}
%g'(k)=k\left( 1- \dfrac{1}{m}\right) ^{k-1} +\dfrac{1}{m}
g'(k) %&=\left( 1- \dfrac{1}{m}\right)^k \ln\left(1-\dfrac{1}{m}\right)+\dfrac{1}{m} \\
> \left( 1- \dfrac{1}{m}\right)\ln\left(1-\dfrac{1}{m}\right)+\dfrac{1}{m}
\end{aligned}
\end{equation}
Let $f(m)=\left( 1- \dfrac{1}{m}\right)\ln\left(1-\dfrac{1}{m}\right)+\dfrac{1}{m}$, we can find $f'(m) = \dfrac{1}{m^2} \ln\left(1-\dfrac{1}{m}\right)< 0$, meaning that the function $f(m)$ is strictly decreasing. When $m$ goes to infinity, we have $\lim\limits_{m \to \infty}  f(m) =  0$. Therefore, we know that $f(m) \geqslant 0$, and $g'(k) > f(m) \geqslant 0$, meaning that the function $g(k)$ is strictly increasing.
%$f(m) \geqslant f(2)=\dfrac{1}{2} - \dfrac{1}{2}\ln\left(\dfrac{1}{2}\right)$. 
We have therefore 
%\begin{equation}
%0<\left( 1-\dfrac{k}{m} \right)   <
%\left( 1-\dfrac{1}{m} \right)  ^k
%\end{equation}
%or 
\begin{equation}
\left( 1-\dfrac{k}{m} \right) ^n   <
\left( 1-\dfrac{1}{m} \right)  ^{kn}
\end{equation}
\end{proof}

%Therefore, the $p'$ of partitioned BF is larger than the $p'$ of BF, this is correct.
%Then the survey says that based on this, the FP probability of partitioned BF is larger than that of BF.
%This derivation is not strict, because two reasons: 1) the FP probability of BF cannot be get by $p'$; 2) the hash mechanism of partitioned BF and BF are different. 
%Therefore, we cannot draw the conclusion that the FP probability of partitioned BF is large according to the above Lemma.


The above lemma shows that $p'_{partition} < p'_{true}$ or equivalently $1-p'_{partition} > 1-p'_{true}$. Thus, we know that with the same parameters, the FP probability of the pBF will be larger than that of the standard BF $f_{true}$, \ie, $f_{partition} > f_{true}$. In addition, Christensen’s bounds in Eq. \ref{fBound} state that the precise value of FP probability for a BF $f_{true}$ is larger than $f_{bloom}$, \ie, $f_{true} > f_{bloom}$. Therefore, we have the following upper and lower bounds:
%%%%%%%%%%%%%%%%%%%%%%%%%%%%%%%%%%%%%%%%%By Qiaobin%%%%%%%%%%%%%%%%%%%%%%%%%%%%%%%

%Having characteristic means that the information entropy is not the maximum, thus the FP probability $f_{partition}$ is larger than the standard Bloom Filter  $f_{true}$, thus we get
%However, Christensen’s bounds in Eq. \ref{fBound} state that the precise value of FP probability for a BF $f_{true}$ is larger than $f_{bloom}$, meaning that


\begin{equation}
\label{bigbig}
f_{partition} > f_{true} > f_{bloom}
\end{equation}



For partitioned BF, the probability that one bit of the array is still 0 $p'$ is shown in Eq. \ref{equ:p'pBF}.
Different from standard Bloom Filter, for partitioned Bloom Filter, the event ``$E(h_1=1),E(h_2=1),E(h_3=1),...,E(h_{i-1}=1)$'' is independent with the event ``$E(h_{i-1}=1)$'', where $E(h_{i-1}=1)$ means the event that the position of $h_{i-1}(x)$ is 1, because each hash function is responsible for one partition, and has no effect with each other. Therefore, 

\begin{equation}
f_{partition}=(1-p'_{partition})^k=\left( 1- \left( 1-\dfrac{k}{m} \right)^n \right) ^k
\end{equation}

Then the formula~\ref{bigbig} becomes

\begin{equation}
\left( 1- \left( 1-\dfrac{k}{m} \right)^n \right) ^k > f_{true} > \left( 1- \left( 1-\dfrac{1}{m} \right)^{nk} \right) ^k
\end{equation}

Then we use the well known limit formula:

\begin{equation}
\lim\limits_{x \to \infty} \left( 1-\dfrac{1}{x}\right) ^{-x} = e
\end{equation}

%and then get

%%%%%%%%%%%%%%%%%%%%%%%%%%%%%%%%%%%%%%%%%By Qiaobin%%%%%%%%%%%%%%%%%%%%%%%%%%%%%%%
%One can easily derive the FP probability for pBF as:
%\begin{equation}
%f_{partition}=(1-p'_{partition})^k=\left( 1- \left( 1-\dfrac{k}{m} \right)^n \right) ^k
%\end{equation}
%Different from standard Bloom Filter, the second independence assumption (the independence between the value of positions mapped by different hash functions) is valid as each hash accesses a bit in a different partition, therefore the above formula is precise resulting in the below upper and lower bounds for $f_{true}$: 

%\emph{Note that the above formula is the correct form of the pBF's FP probability, since all the $k$ hashings access different bits in different partitions, which guarantees that all the $k$ queried bits are independent and satisfies the ``multiple principle''.}
%Therefore, the formula~\ref{bigbig} becomes
%\begin{equation}
%\left( 1- \left( 1-\dfrac{k}{m} \right)^n \right) ^k > f_{true} > \left( 1- \left( 1-\dfrac{1}{m} \right)^{nk} \right) ^k
%\end{equation}
%%%%%%%%%%%%%%%%%%%%%%%%%%%%%%%%%%%%%%%%%By Qiaobin%%%%%%%%%%%%%%%%%%%%%%%%%%%%%%%

Asymptotically when $m$ becomes large, we already know that $f_{bloom}$ converges to the term in Eq. \ref{fBloom}. Nevertheless, the upper bound has also an asymptotic behaviour as :
\begin{equation}
\label{flim}
%\begin{aligned}
\lim\limits_{m \to \infty} \left(1-\left(1-\frac{1}{m}\right)^{nk}\right)^k = \left(1-e^{-nk/m}\right)^k 
%\end{aligned}
\end{equation}
that is the same term as the lower bound limit. Through the Sandwich Theorem (also known as squeeze theorem) we get, similarly to Christensen and to Bose, that :
\begin{equation}
\label{flimtrue}
\lim\limits_{m \to \infty}  f_{true} =  \left(1-e^{-nk/m}\right)^k 
\end{equation}

This means that when $m$ is large, the Bloom's formula can be used with negligible error. However, we still need to evaluate what means \textit{$m$ being large}. We will do this by comparing the two bounds we have in hand: the one from Bose and the one we derived in this paper. 

\begin{figure}
\centering
\includegraphics[width=\figwidth]{LogLog_lowerUpper}
\caption{Upper and lower bound for $f_{true}$ for $k=7$ and $m=10n$.}
\label{fig:uplowbound}
\end{figure}

We show in Figure~\ref{fig:uplowbound} the two upper bounds along with the lower bound obtained for $k=7$  and $m=10n$ as a function of $n$, the number of elements inserted in the BF. As can be seen, the upper bound derived in this paper and the lower bound $f_{bloom}$ converge relatively fast for $n=9$, while the upper bound derived by Bose has a much slower convergence. We can see this better by looking at the behavior of the bounds error ratio $\beta$, defined as $\beta=\frac{upper bound - lower bound}{lower bound}$, for the two bounds in Figure \ref{fig:ratio}.
\begin{figure}
\centering
\includegraphics[width=\figwidth]{LogLog_ratio}
\caption{Bounds error ratio for Bose's bound and the bound derived in this paper for $k=7$ and $m=10n$. }
\label{fig:ratio}
\end{figure}
As can be seen, the gap between the our derived upper bound and $f_{bloom}$ is decreasing polynomially at a constant speed, while Bose bound has a lower speed of convergence. In order to extend this observation, we show in Figure \ref{fig:ratiok} the evolution of the bounds error ratio for a BF with $m=10000$, $n=1000$ and varying $k$. 
\begin{figure}
\centering
\includegraphics[width=\figwidth]{ratioLog_k}
\caption{Bounds error ratio for Bose's bound and the bound derived in this paper for $m=10000$, $n=1000$ and varying $k$.}
\label{fig:ratiok}
\end{figure}
As expected, error involved with using $f_{bloom}$ increases with the number of hash functions $k$ increases. However, it can be seen that the convergence behavior of the bounds derived in this paper is much better than the one obtained by Bose. 



 %\textbf{One Example.}
% \textbf{Example 1:}
%
% For example, $m=2, n=1, k=2$, the result using the Bloom's formula is 4.5/8.
% After hashing, there will be three cases: 10, 01, 11, thus the true false positive is 
%
% \begin{equation}
% \dfrac{1}{4}\times \dfrac{1}{4} +\dfrac{1}{4}\times \dfrac{1}{4} +1 \times \dfrac{2}{4}=\dfrac{5}{8}
% \end{equation} 
% In conclusion, the error of Bloom's formula is 1/16, when $m$ is 2.


% For example, $m=3, \, n=1,  \,  k=2$, there are three cases:
%
% Case I: two hashing map to the same position, the BF array will be one of the three: 100, 010, 001, each with the probability of 1/9. The FP probability of Case I is: 
%
% \begin{equation}
% (\dfrac{1}{3} \times \dfrac{1}{9} ) \times 3 = \dfrac{1}{9}
% \end{equation} 
% Case II: two hashings map to different positions, the BF array will be one of the three: 110, 101, 011, each with the probability of 2/9. First we compute the probability of no false positive:
%
% 1) Both the first and the second hashing positions are 0, the probability is 1/9.
%
% 2) The first hash position is 0, the second hash position is 1, the probability is 
% \begin{equation}
% \dfrac{1}{3} \times \dfrac{2}{3} = \dfrac{2}{9}
% \end{equation} 
%
% 3) The first hash position is 1, the second hash position is 0, the probability is also 2/9.
%
% Therefore, the probability of false positive of Case II is:
%
% \begin{equation}
% 1-\dfrac{1}{9}-\dfrac{2}{9}-\dfrac{2}{9}=\dfrac{4}{9}
% \end{equation}
%
% In sum, the probability of false positive of this example is:
%
%
% \begin{equation}
% \frac{1}{9}\times 3 \times \frac{1}{9}+ \frac{4}{9}\times 3\times \frac{2}{9}=\dfrac{27}{81}
% \end{equation}
%
% Note that the result using Bloom's formula is 25/81, and the error is 2/81.
%
%
%%套公式得出的结果是 25/81,误差是1/40.5
%%
%%\subsection{the results of suboptimal value}
%%For Standard Bloom Filter (SBF), the FP probability is given by:
%%
%%\begin{equation}
%%f=\left(1-\left(1-\dfrac{1}{m}\right)^{nk}\right)^k
%%\end{equation}
%%
%%When m is large, the formula reduces to
%%
%%\begin{equation}
%%\label{math:fpLargem}
%%f=\left(1-e^{-\dfrac{nk}{m}}
%%\right)^k
%%\end{equation} 